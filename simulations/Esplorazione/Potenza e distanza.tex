\documentclass[12pt, a4paper]{report} %standard 12 punti, specifica anche la dimensione del documento
\usepackage[utf8]{inputenc} %specifica l'encoding
\usepackage{graphicx}
\usepackage{float}
\usepackage[section]{placeins}
\usepackage{hyperref}
\hypersetup{
    colorlinks=true,
    linkcolor=blue,
    filecolor=magenta,      
    urlcolor=blue,
    pdftitle={Overleaf Example},
    pdfpagemode=FullScreen,
    }
%\counterwithin*{figure}{section} Posso resettare anche il contatore delle figure a ogni sezione
\counterwithin*{equation}{section} %Resetta il contatore delle equazioni a ogni sezione 
\counterwithin*{equation}{subsection}

\title{Report "Prove potenza e distanza"}
\author{Tommaso Lencioni}
%\date{} % Activate to display a given date or no date (if empty),
         % otherwise the current date is printed 

\begin{document}
\maketitle
In accordo con quanto stabilito durante la call del 18/06 ho progredito nella lettura del codice e ho effettuato nuove prove.\\
\section*{Correzioni rispetto alla versione precedente}
 \begin{itemize}
 	\item Ho rimosso il potere computazionale degli edge device, adesso hanno solo un core senza potenza di calcolo.
 	\item Ho esplorato il codice e ho tracciato a grandi linee la vita di un task e il suo spostarsi tramite il tag di cloudsim.
 	\item Ho individuato la lista di tasks generati ma devo sempre vedere meglio se è effettivamente presente una coda di tasks all'interno del Cloud (purtoppo il codice di CloudSim non è molto chiaro ed è pieno di chiamate a metodi della libreria).
 	\item Come stabilito ho provato ad aumentare la potenza computazionale degi datacenters (sia edge che cloud) e non ho riscontrato un miglioramento nel numero di tasks falliti (mi aspettavo di vedere i fallimenti per delay diminuire).\\
 	Come ci si puo' aspettare il carico sulla CPU è dimezzato.
 	\item Penso di essermi risposto alla domanda "Come mai se non attendo la terminazione dei task la percentuale di successo è maggiore":\\
 	mi pare che cio' avvenga perché in SimulationManager viene rimandato l'evento Print\_log di 10 (secondi di simulazione?) se il flag è attivo e sono stati completati meno task di quelli generati.\\
Se il flag non è attivo viene stampato il log di terminazione.
 	\end{itemize}
\subsection*{Dubbi}
\begin{itemize}
\item Riguardo al fallimento per mobilita' ho controllato il codice e non sembra esserci niente di inaspettato se non un intero "phase" del quale non mi è chiaro l'uso.\\
Pero' continuo ad avere fallimenti di task per mobilita' nonostante quella inserita sia sufficiente a inscrivere l'area di simulazione.
Mantenendo l'area di simulazione 250x250:
\begin{itemize}
	\item Ponendo il range degli edge devices a 180 (circa 250/sqrt(2)) e quello degli edge datacenter a 200 è garantita l'assenza di fallimenti.
	\item Invertendo i due valori si hanno comunque fallimenti per mobilita'.
\end{itemize}
A questo punto non so come la dicitura "coverage area" degli edge datacenter debba essere interpretata.
Nel codice vengono trattate allo stesso modo del range degli edge devices.
\item Devo ancora approfondire la non terminazione della simulazione in caso di core \> 1.\\
Lo faro' nel fine settimana, mi scuso ma ho dovuto sostenere l'orale di Gestione di Reti mercoledì.
\end{itemize}

\end{document}
