\documentclass[12pt, a4paper]{report} %standard 12 punti, specifica anche la dimensione del documento
\usepackage[utf8]{inputenc} %specifica l'encoding
\usepackage{graphicx}
\usepackage{float}
\usepackage[section]{placeins}
\usepackage{hyperref}
\hypersetup{
    colorlinks=true,
    linkcolor=blue,
    filecolor=magenta,      
    urlcolor=blue,
    pdftitle={Overleaf Example},
    pdfpagemode=FullScreen,
    }
%\counterwithin*{figure}{section} Posso resettare anche il contatore delle figure a ogni sezione
\counterwithin*{equation}{section} %Resetta il contatore delle equazioni a ogni sezione 
\counterwithin*{equation}{subsection}

\title{Leader election and offload}
\author{Tommaso Lencioni}
%\date{} % Activate to display a given date or no date (if empty),
         % otherwise the current date is printed 

\begin{document}
%\maketitle
\section*{Idea}
\begin{itemize}
	\item An edge data center that is also an orchestrator should be able to elect a leader among the other edge data centers.
	\item A leader (passively known to cover such a role by not having a leader) can receive tasks and offload them to its subjecteds.
	\item If there is no computational power left the task is offloaded on the Cloud.
\end{itemize}

\section*{Implementation}
\begin{itemize}
 	\item I started by creating a custom class that extends 	DefaultDataCenter using the proposed ClusterEdgeDevice as a base.
 	\item In startInternal a task with tag LEADER\_ELECTION is scheduled with INITIALIZATION\_TIME + 1 delay.
 	\item In process event the custom tag is caught.
 	\item There is a check for:
 	\begin{itemize}
 		\item the device is an edge data center,
 		\item the device is an orchestrator,
 		\item "LEADER" is the orchestration method,
  	\end{itemize}
  	\item If everything is true then the method \textbf{leader} is called.
  	\begin{itemize}
 		\item in a loop every datacenter is taken into account and is checked whether it:
 		\begin{itemize}
 			\item is not the same data center as the one evaluating,
 			\item is an edge data center
 			\item the distance between the two data centers is smaller than the range of the edge data centers.
 		\end{itemize}
 		\item If it is the case then there is an evaluation regarding the MIPS (I used that criterion for electing a leader).
 		\begin{itemize}
 			\item If the MIPS of the candidate are greater than the evaluator
 			\item and the max of the data center seen until that point is lower than the MIPS of the candidate
 		\end{itemize}
 		\item then the candidate becomes the (potentially temporary) leader, the max is set to leader's MIPS.
 		\item Upon ending the loop if a leader has been found the current datacenter is added to its subjected list. This will be used when a new task must be executed by someone with enough free computational power.
  	\end{itemize}
  	
\end{itemize}


\end{document}
